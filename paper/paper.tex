% Options for packages loaded elsewhere
\PassOptionsToPackage{unicode}{hyperref}
\PassOptionsToPackage{hyphens}{url}
\PassOptionsToPackage{dvipsnames,svgnames,x11names}{xcolor}
%
\documentclass[
  letterpaper,
  DIV=11,
  numbers=noendperiod]{scrartcl}

\usepackage{amsmath,amssymb}
\usepackage{iftex}
\ifPDFTeX
  \usepackage[T1]{fontenc}
  \usepackage[utf8]{inputenc}
  \usepackage{textcomp} % provide euro and other symbols
\else % if luatex or xetex
  \usepackage{unicode-math}
  \defaultfontfeatures{Scale=MatchLowercase}
  \defaultfontfeatures[\rmfamily]{Ligatures=TeX,Scale=1}
\fi
\usepackage{lmodern}
\ifPDFTeX\else  
    % xetex/luatex font selection
\fi
% Use upquote if available, for straight quotes in verbatim environments
\IfFileExists{upquote.sty}{\usepackage{upquote}}{}
\IfFileExists{microtype.sty}{% use microtype if available
  \usepackage[]{microtype}
  \UseMicrotypeSet[protrusion]{basicmath} % disable protrusion for tt fonts
}{}
\makeatletter
\@ifundefined{KOMAClassName}{% if non-KOMA class
  \IfFileExists{parskip.sty}{%
    \usepackage{parskip}
  }{% else
    \setlength{\parindent}{0pt}
    \setlength{\parskip}{6pt plus 2pt minus 1pt}}
}{% if KOMA class
  \KOMAoptions{parskip=half}}
\makeatother
\usepackage{xcolor}
\setlength{\emergencystretch}{3em} % prevent overfull lines
\setcounter{secnumdepth}{5}
% Make \paragraph and \subparagraph free-standing
\ifx\paragraph\undefined\else
  \let\oldparagraph\paragraph
  \renewcommand{\paragraph}[1]{\oldparagraph{#1}\mbox{}}
\fi
\ifx\subparagraph\undefined\else
  \let\oldsubparagraph\subparagraph
  \renewcommand{\subparagraph}[1]{\oldsubparagraph{#1}\mbox{}}
\fi


\providecommand{\tightlist}{%
  \setlength{\itemsep}{0pt}\setlength{\parskip}{0pt}}\usepackage{longtable,booktabs,array}
\usepackage{calc} % for calculating minipage widths
% Correct order of tables after \paragraph or \subparagraph
\usepackage{etoolbox}
\makeatletter
\patchcmd\longtable{\par}{\if@noskipsec\mbox{}\fi\par}{}{}
\makeatother
% Allow footnotes in longtable head/foot
\IfFileExists{footnotehyper.sty}{\usepackage{footnotehyper}}{\usepackage{footnote}}
\makesavenoteenv{longtable}
\usepackage{graphicx}
\makeatletter
\def\maxwidth{\ifdim\Gin@nat@width>\linewidth\linewidth\else\Gin@nat@width\fi}
\def\maxheight{\ifdim\Gin@nat@height>\textheight\textheight\else\Gin@nat@height\fi}
\makeatother
% Scale images if necessary, so that they will not overflow the page
% margins by default, and it is still possible to overwrite the defaults
% using explicit options in \includegraphics[width, height, ...]{}
\setkeys{Gin}{width=\maxwidth,height=\maxheight,keepaspectratio}
% Set default figure placement to htbp
\makeatletter
\def\fps@figure{htbp}
\makeatother
\newlength{\cslhangindent}
\setlength{\cslhangindent}{1.5em}
\newlength{\csllabelwidth}
\setlength{\csllabelwidth}{3em}
\newlength{\cslentryspacingunit} % times entry-spacing
\setlength{\cslentryspacingunit}{\parskip}
\newenvironment{CSLReferences}[2] % #1 hanging-ident, #2 entry spacing
 {% don't indent paragraphs
  \setlength{\parindent}{0pt}
  % turn on hanging indent if param 1 is 1
  \ifodd #1
  \let\oldpar\par
  \def\par{\hangindent=\cslhangindent\oldpar}
  \fi
  % set entry spacing
  \setlength{\parskip}{#2\cslentryspacingunit}
 }%
 {}
\usepackage{calc}
\newcommand{\CSLBlock}[1]{#1\hfill\break}
\newcommand{\CSLLeftMargin}[1]{\parbox[t]{\csllabelwidth}{#1}}
\newcommand{\CSLRightInline}[1]{\parbox[t]{\linewidth - \csllabelwidth}{#1}\break}
\newcommand{\CSLIndent}[1]{\hspace{\cslhangindent}#1}

\KOMAoption{captions}{tableheading}
\makeatletter
\makeatother
\makeatletter
\makeatother
\makeatletter
\@ifpackageloaded{caption}{}{\usepackage{caption}}
\AtBeginDocument{%
\ifdefined\contentsname
  \renewcommand*\contentsname{Table of contents}
\else
  \newcommand\contentsname{Table of contents}
\fi
\ifdefined\listfigurename
  \renewcommand*\listfigurename{List of Figures}
\else
  \newcommand\listfigurename{List of Figures}
\fi
\ifdefined\listtablename
  \renewcommand*\listtablename{List of Tables}
\else
  \newcommand\listtablename{List of Tables}
\fi
\ifdefined\figurename
  \renewcommand*\figurename{Figure}
\else
  \newcommand\figurename{Figure}
\fi
\ifdefined\tablename
  \renewcommand*\tablename{Table}
\else
  \newcommand\tablename{Table}
\fi
}
\@ifpackageloaded{float}{}{\usepackage{float}}
\floatstyle{ruled}
\@ifundefined{c@chapter}{\newfloat{codelisting}{h}{lop}}{\newfloat{codelisting}{h}{lop}[chapter]}
\floatname{codelisting}{Listing}
\newcommand*\listoflistings{\listof{codelisting}{List of Listings}}
\makeatother
\makeatletter
\@ifpackageloaded{caption}{}{\usepackage{caption}}
\@ifpackageloaded{subcaption}{}{\usepackage{subcaption}}
\makeatother
\makeatletter
\@ifpackageloaded{tcolorbox}{}{\usepackage[skins,breakable]{tcolorbox}}
\makeatother
\makeatletter
\@ifundefined{shadecolor}{\definecolor{shadecolor}{rgb}{.97, .97, .97}}
\makeatother
\makeatletter
\makeatother
\makeatletter
\makeatother
\ifLuaTeX
  \usepackage{selnolig}  % disable illegal ligatures
\fi
\IfFileExists{bookmark.sty}{\usepackage{bookmark}}{\usepackage{hyperref}}
\IfFileExists{xurl.sty}{\usepackage{xurl}}{} % add URL line breaks if available
\urlstyle{same} % disable monospaced font for URLs
\hypersetup{
  pdftitle={Survival Compass: Statistical Insights into Lung Cancer Patients' Journey Post-Diagnosis},
  pdfauthor={Lexi Knight},
  colorlinks=true,
  linkcolor={blue},
  filecolor={Maroon},
  citecolor={Blue},
  urlcolor={Blue},
  pdfcreator={LaTeX via pandoc}}

\title{Survival Compass: Statistical Insights into Lung Cancer Patients'
Journey Post-Diagnosis\thanks{Code and data are available at:
https://github.com/LexiKnight/Lung\_Cancer/tree/main}}
\author{Lexi Knight}
\date{April 4, 2024}

\begin{document}
\maketitle
\begin{abstract}
This study investigates the impact of pathogenic stage and treatment
modalities on lung cancer survival post-diagnosis. Analysis of patient
data reveals significant correlation between pathogenic stage, seeking
treatment and survival outcomes. Notably, patients at advanced stages
with metastases in distant sites beyond the lung, extensive lymph node
involvement and tumors with extensive growth, invading nearby structures
demonstrate lower survival rates. These findings underscore the critical
importance of early detection, tailored treatment strategies and ongoing
research efforts to enhance lung cancer survival rates globally.
\end{abstract}
\ifdefined\Shaded\renewenvironment{Shaded}{\begin{tcolorbox}[sharp corners, borderline west={3pt}{0pt}{shadecolor}, boxrule=0pt, interior hidden, breakable, frame hidden, enhanced]}{\end{tcolorbox}}\fi

\hypertarget{introduction}{%
\section{Introduction}\label{introduction}}

Clinging to life amidst the shadows of lung cancer, where every breath
becomes a battleground. Survival becomes not just a statistic but an
interplay between several individual characteristics. We explore the
hidden keys to defying the odds and emerging victorious against one of
the deadliest adversaries of our time. Lung cancer is the leading cause
of cancer-related deaths in the world (Park, 2017). It is a disease that
develops in the lining of the airways in lung tissues. Non-small cell
lung cancer (NSCLC) is the most common type, accounting for 80-85\% of
all lung cancers according to the American Cancer society (Markman,
2023). Staging is important for prognosis and making treatment
decisions. Common treatments include surgery, radiation therapy and
chemotherapy (Kai, 2021). Pathogenic stage is determined by presence of
nearby metastasis, lymph node involvement as well as tumor spread and
size (Markman, 2023). This paper investigates the relationship between
lung cancer patients' survival and pathogenic stage. The estimand is the
median survival time in days post-diagnosis. We also look at whether
patients decided to have treatment and if so, which method; radiation
therapy or chemotherapy. Through analysis of a dataset made up of 981
patients in Sydney, Australia, we offer insight into the prognostic
markers.

Tumor size is often the main determinant of stage and treatment. As
tumor categories increase, the tumor expands, invading nearby structures
(Zhang, 2015). A study involving 52,287 patients diagnosed between the
years 1998 and 2003 found tumor size to be an independent prognostic
factor in estimating overall survival. The authors found that patients
presenting with larger tumors predicted a worse prognosis and thus are
associated with a decrease in survival. There is a similar relationship
between extensive lymph node involvement and patient survival (Zhang,
2015). Initial spread of cancer cells are localized, then become
regional, involving nearby lymph nodes and the most severe cases
comprises expansion to other organs such as the brain, liver and bones
(Markman, 2023). A study looked at five year survival rates based on the
severity of spread. 62.8\% of patients with localized spread, 34.8\% of
patients with regional and 8\% of patients with distant, advanced spread
were found to survive for 5 years post diagnosis. More than half of
these lung cancer patients have advanced spread to other organs when
diagnosed(Markman, 2023). Overall, it is found that patients with no
regional lymph node metastases, and smaller tumors are easier to be
treated and thus are associated with improved survival rates (Zhang,
2015).

Presence of metastatic LN is one of the most important determinants of
prognosis of NSCLC cases (Kai, 2021). In the early stage, cancer has not
spread to lymph nodes. As severity increases, lymph node metastasis
sequentially spreads to more distant lymph nodes such as mediastinal and
there is severe lymph node involvement (Park, 2017). Lymph node
involvement, also termed lymph node ratio, is a crucial factor in
guiding treatment options (Kai, 2021). A study made up of 97 patients
with a mean age of 63 who have undergone surgery between the years 2009
and 2015 in Korea find that increased lymph node involvement is
associated with a more advanced disease status and hence affiliated with
prognosis (Park, 2017). Another study looked at 11,341 NSCLC patients
between the years 2004 to 2015, from 18 geographically diverse
populations, covering approximately 28\% of the population of the United
States. These patients were treatment naive and underwent surgical
resection of the tumor. Although 5757 patients died, the rest showed
great results, with a median survival of 22 months (Kai, 2021). The
authors found that patients with low lymph node involvement lead to
higher survival compared to patients with high lymph node ratios. A
regression analysis revealed that lymph node ratio is an independent and
significant predictor of patient survival. The authors also observed
that disease burden and anatomical location of the lymph nodes involved
may influence the patients survival (Kai, 2021).

After tumor size, LN involvement and presence of distant metastasis are
categorized, the pathogenic stage of the cancer is then determined
(Eldridge, 2022). The most valuable prognostic factor in non-small cell
lung cancer is the pathogenic stage (Park, 2017) .Stage is determined by
tumor size, number of tumors and where the cancer has spread. Stage 1 is
localized spread, stage 2 and 3 is regional spread while stage 4 is
distant spread of the tumor (Eldridge, 2022). Cancer stage was
determined using the seventh American Joint Committee on Cancer staging
system (AJCC) (Park, 2017). A study done in Australia including 2119
lung cancer patients illustrated those with stage IV disease, the most
advanced stage, showed shorter survival than those at lower stages
(Denton, 2016). The earlier the cancer is found, that is the lower the
pathogenic stage, the greater the likelihood curative radiation therapy
is an effective treatment (Eldridge, 2022). However, there is minimal
literature looking at post-diagnosis survival rates based on pathogenic
stage and method of treatment. The extent of this disease illustrates
the importance of living a healthy lifestyle, undergoing regular
screening and development of improved treatment methods. Over the past
decade, there has been great improvement of lymph node assessment in
cancer patients (Kai, 2021). Experts hope survival rates continue to
improve with new therapies and treatment approaches (Markman, 2023).

The remainder of this paper is structured as follows. In
Section~\ref{sec-data}, we visualize the exploration of variables
constituting the pathogenic stage and treatment types.
Section~\ref{sec-model}, outlines the model employed to analyze the
relationship between these variables and the duration of survival
post-diagnosis. Moreover, Section~\ref{sec-results} offers visual
depictions of the study's outcomes. Finally, in
Section~\ref{sec-discussion}, we summarize the primary findings, propose
avenues for enhancement, and identify potential areas for future
research.

\hypertarget{sec-data}{%
\section{Data}\label{sec-data}}

Our data is

\begin{verbatim}
Warning: package 'ggplot2' was built under R version 4.2.3
\end{verbatim}

\begin{verbatim}
Warning: package 'dplyr' was built under R version 4.2.3
\end{verbatim}

\begin{verbatim}

Attaching package: 'dplyr'
\end{verbatim}

\begin{verbatim}
The following objects are masked from 'package:stats':

    filter, lag
\end{verbatim}

\begin{verbatim}
The following objects are masked from 'package:base':

    intersect, setdiff, setequal, union
\end{verbatim}

\begin{verbatim}
Warning: package 'showtext' was built under R version 4.2.3
\end{verbatim}

\begin{verbatim}
Loading required package: sysfonts
\end{verbatim}

\begin{verbatim}
Warning: package 'sysfonts' was built under R version 4.2.3
\end{verbatim}

\begin{verbatim}
Loading required package: showtextdb
\end{verbatim}

\begin{verbatim}
Warning: package 'showtextdb' was built under R version 4.2.3
\end{verbatim}

\begin{verbatim}
Warning: package 'arrow' was built under R version 4.2.3
\end{verbatim}

\begin{verbatim}

Attaching package: 'arrow'
\end{verbatim}

\begin{verbatim}
The following object is masked from 'package:utils':

    timestamp
\end{verbatim}

\begin{figure}

{\centering \includegraphics{paper_files/figure-pdf/fig-bills-1.pdf}

}

\caption{\label{fig-bills-1}bb}

\end{figure}

\begin{figure}

{\centering \includegraphics{paper_files/figure-pdf/fig-bills-2.pdf}

}

\caption{\label{fig-bills-2}bb}

\end{figure}

\begin{figure}

{\centering \includegraphics{paper_files/figure-pdf/fig-bills-3.pdf}

}

\caption{\label{fig-bills-3}bb}

\end{figure}

\begin{figure}

{\centering \includegraphics{paper_files/figure-pdf/fig-bills-4.pdf}

}

\caption{\label{fig-bills-4}bb}

\end{figure}

\begin{figure}

{\centering \includegraphics{paper_files/figure-pdf/fig-bills-5.pdf}

}

\caption{\label{fig-bills-5}bb}

\end{figure}

\begin{figure}

{\centering \includegraphics{paper_files/figure-pdf/fig-bills-6.pdf}

}

\caption{\label{fig-bills-6}bb}

\end{figure}

Talk more about it.

\hypertarget{sec-model}{%
\section{Model}\label{sec-model}}

\hypertarget{linear-regression-model}{%
\subsection{Linear Regression Model}\label{linear-regression-model}}

\hypertarget{interested-in-predicting-days_to_death-a-continuous-outcome-variable-based-on}{%
\section{interested in predicting days\_to\_death, a continuous outcome
variable based
on}\label{interested-in-predicting-days_to_death-a-continuous-outcome-variable-based-on}}

\hypertarget{multiple-predictor-variables-pathogenic_stage-lymph_node_involvement}{%
\section{multiple predictor variables (pathogenic\_stage,
lymph\_node\_involvement,}\label{multiple-predictor-variables-pathogenic_stage-lymph_node_involvement}}

\hypertarget{presence_of_distant_metastasis-tumor_size-treatment_decision-treatment_type}{%
\section{presence\_of\_distant\_metastasis, tumor\_size,
treatment\_decision,
treatment\_type)}\label{presence_of_distant_metastasis-tumor_size-treatment_decision-treatment_type}}

\hypertarget{linear-regression-model-are-suitable-for-predicting-continuous-outcomes}{%
\section{Linear regression model are suitable for predicting continuous
outcomes}\label{linear-regression-model-are-suitable-for-predicting-continuous-outcomes}}

\hypertarget{quantifying-relationship-between-predictor-variables-and-outcome-variable.}{%
\section{quantifying relationship between predictor variables and
outcome
variable.}\label{quantifying-relationship-between-predictor-variables-and-outcome-variable.}}

\hypertarget{fit-a-linear-regression-model-to-predict-continuous-outcome-variable}{%
\section{Fit a linear regression model to predict continuous outcome
variable}\label{fit-a-linear-regression-model-to-predict-continuous-outcome-variable}}

\hypertarget{days_to_death}{%
\section{`days\_to\_death'}\label{days_to_death}}

\hypertarget{model-set-up}{%
\subsection{Model set-up}\label{model-set-up}}

In this study, we aim to investigate the survival outcomes of lung
cancer patients post-diagnosis. Let \(y_i\) denote the number of days
from diagnosis to death for patient \(i\). We consider several
predictors including pathogenic stage, lymph node involvement, presence
of distant metastasis, tumor size, and treatment type.

We model the survival time \(y_i\) using a linear regression framework:

y\_i\textbar{}\mu\_i, \sigma \&\sim \mbox{Normal}(\mu\_i, \sigma)
\textbackslash{}

where \(mi_i\) is the expected survival time for patient \(i\), and
\(σ\) represents the standard deviation of the survival times. The
linear predictor \(mu_i\) is specified as:

\mu\emph{i = \alpha + \beta}\{\text{pathologic\_stage}\}
\times \text{pathologic\_stage}\emph{i + \beta}\{\text{lymph\_node}\}
\times \text{lymph\_node\_involvement}\emph{i +
\beta}\{\text{metastasis}\}
\times \text{presence\_of\_distant\_metastasis}\emph{i +
\beta}\{\text{tumor\_size}\} \times \text{tumor\_size}\emph{i +
\beta}\{\text{treatment\_type}\} \times \text{treatment\_type}\_i

Here, \alpha represents the intercept term, \beta coefficients denote
the effect of respective predictors, and subscripts \(i\) denote
individual patients.

Priors are specified for the model parameters as follows:

\begin{align*}
\alpha &\sim \text{Normal}(0, 2.5) \\
\beta_{\text{pathologic_stage}}, \beta_{\text{lymph_node}}, \beta_{\text{metastasis}}, \beta_{\text{tumor_size}}, \beta_{\text{treatment_type}} &\sim \text{Normal}(0, 2.5) \\
\sigma &\sim \text{Exponential}(1)
\end{align*}

TEMPLATE: Define \(y_i\) as the number of seconds that the plane
remained aloft. Then \(\beta_i\) is the wing width and \(\gamma_i\) is
the wing length, both measured in millimeters.

\begin{align} 

\mu_i &= \alpha + \beta_i + \gamma_i\\
\alpha &\sim \mbox{Normal}(0, 2.5) \\
\beta &\sim \mbox{Normal}(0, 2.5) \\
\gamma &\sim \mbox{Normal}(0, 2.5) \\
\sigma &\sim \mbox{Exponential}(1)
\end{align}

We run the model in R (R Core Team 2023) using the \texttt{rstanarm}
package of Goodrich et al. (2022). We use the default priors from
\texttt{rstanarm}.

\hypertarget{model-justification}{%
\subsubsection{Model justification}\label{model-justification}}

We anticipate that the survival time of lung cancer patients
post-diagnosis will be influenced by various clinical factors such as
pathologic stage, extent of lymph node involvement, presence of distant
metastasis, tumor size, and treatment type. Specifically, we expect that
advanced pathologic stages, increased lymph node involvement, presence
of distant metastasis, larger tumor sizes, and certain treatment types
will be associated with shorter survival times.

The linear regression model allows us to quantify the relationships
between these predictors and survival outcomes, providing valuable
insights into the factors influencing the prognosis of lung cancer
patients.

\hypertarget{sec-results}{%
\section{Results}\label{sec-results}}

Our results are summarized in @.

\includegraphics{paper_files/figure-pdf/unnamed-chunk-4-1.pdf}

\includegraphics{paper_files/figure-pdf/unnamed-chunk-4-2.pdf}

\includegraphics{paper_files/figure-pdf/unnamed-chunk-4-3.pdf}

\includegraphics{paper_files/figure-pdf/unnamed-chunk-4-4.pdf}

\includegraphics{paper_files/figure-pdf/unnamed-chunk-4-5.pdf}

\includegraphics{paper_files/figure-pdf/unnamed-chunk-4-6.pdf}

\includegraphics{paper_files/figure-pdf/unnamed-chunk-4-7.pdf}

\includegraphics{paper_files/figure-pdf/unnamed-chunk-4-8.pdf}

\includegraphics{paper_files/figure-pdf/unnamed-chunk-4-9.pdf}

\includegraphics{paper_files/figure-pdf/unnamed-chunk-4-10.pdf}

\includegraphics{paper_files/figure-pdf/unnamed-chunk-4-11.pdf}

\includegraphics{paper_files/figure-pdf/unnamed-chunk-4-12.pdf}

\includegraphics{paper_files/figure-pdf/unnamed-chunk-4-13.pdf}

\includegraphics{paper_files/figure-pdf/unnamed-chunk-4-14.pdf}

\includegraphics{paper_files/figure-pdf/unnamed-chunk-4-15.pdf}

\includegraphics{paper_files/figure-pdf/unnamed-chunk-4-16.pdf}

\includegraphics{paper_files/figure-pdf/unnamed-chunk-4-17.pdf}

\begin{table}

\end{table}

\hypertarget{sec-discussion}{%
\section{Discussion}\label{sec-discussion}}

\hypertarget{sec-first-point}{%
\subsection{First discussion point}\label{sec-first-point}}

\hypertarget{second-discussion-point}{%
\subsection{Second discussion point}\label{second-discussion-point}}

\hypertarget{third-discussion-point}{%
\subsection{Third discussion point}\label{third-discussion-point}}

\hypertarget{weaknesses-and-next-steps}{%
\subsection{Weaknesses and next steps}\label{weaknesses-and-next-steps}}

\newpage

\appendix

\hypertarget{sec-appendix}{%
\section{Appendix}\label{sec-appendix}}

\hypertarget{additional-data-details}{%
\section{Additional data details}\label{additional-data-details}}

\hypertarget{sec-model-details}{%
\section{Model details}\label{sec-model-details}}

we compare the posterior with the prior. This shows\ldots{}

\begin{figure}

\begin{minipage}[t]{0.50\linewidth}

{\centering 

Examining how the model fits, and is affected by, the data

}

\end{minipage}%

\caption{\label{fig-ppcheckandposteriorvsprior}\textbf{?(caption)}}

\end{figure}

\hypertarget{diagnostics}{%
\subsection{Diagnostics}\label{diagnostics}}

Is this needed?

\begin{figure}

\begin{minipage}[t]{0.50\linewidth}

{\centering 

Checking the convergence of the MCMC algorithm

}

\end{minipage}%

\caption{\label{fig-stanareyouokay}\textbf{?(caption)}}

\end{figure}

\newpage

\hypertarget{references}{%
\section*{References}\label{references}}
\addcontentsline{toc}{section}{References}

\hypertarget{refs}{}
\begin{CSLReferences}{1}{0}
\leavevmode\vadjust pre{\hypertarget{ref-rstanarm}{}}%
Goodrich, Ben, Jonah Gabry, Imad Ali, and Sam Brilleman. 2022.
{``Rstanarm: {Bayesian} Applied Regression Modeling via {Stan}.''}
\url{https://mc-stan.org/rstanarm/}.

\leavevmode\vadjust pre{\hypertarget{ref-citeR}{}}%
R Core Team. 2023. \emph{R: A Language and Environment for Statistical
Computing}. Vienna, Austria: R Foundation for Statistical Computing.
\url{https://www.R-project.org/}.

\end{CSLReferences}



\end{document}
